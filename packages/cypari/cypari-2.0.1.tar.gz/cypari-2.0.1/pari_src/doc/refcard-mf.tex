% Copyright (c) 2007-2016 Karim Belabas.
% Permission is granted to copy, distribute and/or modify this document
% under the terms of the GNU General Public License

% Reference Card for PARI-GP, Algebraic Number Theory.
% Author:
%  Karim Belabas
%  Universite de Bordeaux, 351 avenue de la Liberation, F-33405 Talence
%  email: Karim.Belabas@math.u-bordeaux.fr
%
% See refcard.tex for acknowledgements and thanks.
\def\TITLE{Modular forms, modular symbols}
\input refmacro.tex

\section{Modular Forms}

To be completed later.
\bigskip

\section{Modular Symbols}
Let $G = \Gamma_0(N)$, $V_k = \QQ[X,Y]_{k-2}$. We let
$\Delta = \text{Div}^0(\PP^1(\QQ))$; an element of $\Delta$ is a \emph{path}
between cusps of $X_0(N)$ via the identification $[b]-[a] \to $ the path from
$a$ to $b$. A path is coded by the pair $[a,b]$, where $a,b$ are rationals
or \kbd{oo}, denoting the point at infinity $(1:0)$.\hfil\break

Let $\MM_k(G) = \Hom_G(\Delta, V_k) \simeq H^1_c(X_0(G),V_k)$;
an element of $\MM_k(G)$ is a $V_k$-valued \emph{modular symbol}.
There is a natural decomposition $\MM_k(G) = \MM_k(G)^+ \oplus \MM_k(G)^-$
under the action of the $*$ involution, induced by complex conjugation.
The \kbd{msinit} function computes either $\MM_k$ ($\varepsilon = 0$) or
its $\pm$-parts ($\varepsilon = \pm1$) and fixes a minimal
set of $\ZZ[G]$-generators $(g_i)$ of $\Delta$.\hfil\break

\li{initialize $M = \MM_k(\Gamma_0(N))^\varepsilon$}
   {msinit$(N,k,\{\varepsilon=0\})$}
\li{the level $M$}{msgetlevel$(M)$}
\li{the weight $k$}{msgetweight$(M)$}
\li{the sign $\varepsilon$}{msgetsign$(M)$}
\smallskip

\li{$\ZZ[G]$-generators and relations for $\Delta$}{mspathgens$(M)$}
\li{Decompose $p = [a,b]$ on the $(g_i)$}{mspathlog$(M,p)$}
\smallskip

\subsec{Create a symbol}
\li{Eisenstein symbol attached to cusp $c$}{msfromcusp$(M,c)$}
\li{Cuspidal symbol attached to $E/\QQ$}{msfromell$(E)$}
\li{symbol having given Hecke eigenvalues}{msfromhecke$(M,v,\{H\})$}
\li{is $s$ a symbol ?}{msissymbol$(M,s)$}
\li{the list of all $s(g_i)$}{mseval$(M,s)$}
\li{evaluate symbol $s$ on path $p=[a,b]$}{mseval$(M,s,p)$}

\subsec{Operators}
An operator is given by a matrix of a fixed $\QQ$-basis. $H$, if given, is a
stable $\QQ$-subspace of $\MM_k(G)$: operator is restricted to $H$.\hfil\break
\li{matrix of Hecke operator $T_p$ or $U_p$}{mshecke$(M,p,\{H\})$}
\li{matrix of Atkin-Lehner $w_Q$}{msatkinlehner$(M,Q\{H\})$}
\li{matrix of the $*$ involution}{msstar$(M,\{H\})$}

\subsec{Subspaces} A subspace is given by a structure allowing
quick projection and restriction of linear operators. Its fist
component is a matrix with integer coefficients whose columns for a
$\QQ$-basis. If $H$ is a Hecke-stable subspace of $M_k(G)^+$ or $M_k(G)^-$,
it can be split into a direct sum of Hecke-simple subspaces.
To a simple subspace corresponds a single normalized newform
$\sum_n a_n q^n$.
\hfil\break
\li{cuspidal subspace $S_k(G)^\varepsilon$}{mscuspidal$(M)$}
\li{Eisenstein subspace $E_k(G)^\varepsilon$}{mseisenstein$(M)$}
\li{new part of $S_k(G)^\varepsilon$}{msnew$(M)$}
\li{split $H$ into simple subspaces (of dim $\leq d$)}{mssplit$(M,H,\{d\})$}
\li{$(a_1,\dots, a_B)$ for attached newform}
   {msqexpansion$(M, H, \{B\})$}

\medskip
\subsec{Overconvergent symbols and $p$-adic $L$ functions}
Let $M$ be a full modular symbol space given by \kbd{msinit}
and $p$ be a prime. To a classical modular symbol $\phi$ of level $N$
($v_p(N)\leq 1$), which is an eigenvector for $T_p$ with non-zero eigenvalue
$a_p$, we can attach a $p$-adic $L$-function $L_p$. The function $L_p$
is defined on continuous characters of $\text{Gal}(\QQ(\mu_{p^\infty})/\QQ)$;
in GP we allow characters $\langle \chi \rangle^{s_1} \tau^{s_2}$, where
$(s_1,s_2)$ are integers, $\tau$ is the Teichm\"uller character and $\chi$ is
the cyclotomic character.

The symbol $\phi$ can be lifted to an \emph{overconvergent} symbol $\Phi$,
taking values in spaces of $p$-adic distributions (represented in GP by a
list of moments modulo $p^n$).

\kbd{mspadicinit} precomputes data used to lift symbols. If $\fl$ is given,
it speeds up the computation by assuming that $v_p(a_p) = 0$ if $\fl = 0$
(fastest), and that $v_p(a_p) \geq \fl$ otherwise (faster as $\fl$ increases).

\kbd{mspadicmoments} computes distributions \var{mu} attached to $\Phi$
allowing to compute $L_p$ to high accuracy.

\li{initialize $\var{Mp}$ to lift symbols}{mspadicinit$(M,p,n,\{\fl\})$}
\li{lift symbol $\phi$}{mstooms$(\var{Mp}, \phi)$}
\li{eval overconvergent symbol $\Phi$ on path $p$}{msomseval$(\var{Mp},\Phi,p)$}
\li{\var{mu} for $p$-adic $L$-functions}
   {mspadicmoments$(\var{Mp}, S, \{D=1\})$}
\li{$L_p^{(r)}(\chi^s)$, $s = [s_1,s_2]$}
   {mspadicL$(\var{mu}, \{s = 0\}, \{r = 0\})$}
\li{$\hat{L}_p(\tau^i)(x)$}{mspadicseries$(\var{mu}, \{i = 0\})$}

\copyrightnotice
\bye
