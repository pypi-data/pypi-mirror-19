\documentclass[briefvorlage,
   parskip=half,%%           Abstand zwischen Absaetzen
   fromfax=on,%%             Faxnummer!
   fromemail=on,%%           Emailadresse!
   locfield=wide,%%          zusaetzliches Feld fuer Absender!
   draft=false%%             Entwurfsmodus!
]{scrlttr2}

\usepackage{advdate} % fuer die Fristberechnung
% Bei Briefen in das Ausland
%\setkomavar{fromcountry}{Germany}


% Falls eine Unterschrift genutzt werden soll
%\setkomavar{signature}{\\[-3\baselineskip]
%\includegraphics[scale=.5]{../signature.pdf}\\
%Foo Bar}

\begin{document}
 \begin{letter}{
Company\\
Street X\\
XXXXX Berlin}

\setkomavar{subject}{Auskunft und Widerruf der Genehmigung zur Speicherung und Nutzung meiner Daten für werbliche Zwecke}

%\setkomavar{customer}[\normalfont Kundennummer]{}
%\setkomavar{yourref}[\normalfont Ihr Zeichen] {Foo}
\setkomavar{myref}[\normalfont Mein Zeichen]{2342}
\setkomavar{date}[\normalfont Datum]{\today}
\opening{Sehr geehrte Damen und Herren,}

Sie haben mir unverlangt Werbung zugeschickt. Dies möchte ich zum Anlass nehmen, die Nutzung von Daten mit Ihnen zu regeln.

Ich fordere Sie auf, mir folgende Auskünfte zu erteilen:

\begin{itemize}
\item Über welche gespeicherten Daten zu meiner Person verfügen Sie? Woher haben Sie diese Daten?

\item An welche Empfänger oder sonstige Stellen werden diese Daten weitergegeben?

\item Zu welchem Zweck erfolgt diese Speicherung?

\end{itemize}

Ich widerspreche der Nutzung und Übermittlung meiner Daten für Zwecke der Werbung oder der Markt- oder Meinungsforschung. Sie sind daher verpflichtet, die Daten unverzüglich für diese Zwecke zu sperren. Ich bitte Sie um eine schriftliche Bestätigung der Sperrung.

Ich setze Ihnen zur Erfüllung meiner Forderungen eine Frist bis zum \DayAfter[21]. Sollten Sie dieses Schreiben ignorieren, werde ich mich an den zuständigen Landesdatenschutzbeauftragten wenden.

\closing{Mit freundlichen Grüßen}
\end{letter}

 
\end{document}
