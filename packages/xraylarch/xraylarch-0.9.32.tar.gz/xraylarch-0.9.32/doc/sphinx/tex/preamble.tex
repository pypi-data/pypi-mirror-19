\submitted{November 2012}
\copyrightyear{2012}
\adviser{Professor John P. Smith}
\department{Computer Science}

\newcommand{\proquestmode}{}

\abstract{
\input{abstract.inc}
}

\acknowledgements{
%% Copyright 2008, The TPIE development team
%% 
%% This file is part of TPIE.
%% 
%% TPIE is free software: you can redistribute it and/or modify it under
%% the terms of the GNU Lesser General Public License as published by the
%% Free Software Foundation, either version 3 of the License, or (at your
%% option) any later version.
%% 
%% TPIE is distributed in the hope that it will be useful, but WITHOUT ANY
%% WARRANTY; without even the implied warranty of MERCHANTABILITY or
%% FITNESS FOR A PARTICULAR PURPOSE.  See the GNU Lesser General Public
%% License for more details.
%% 
%% You should have received a copy of the GNU Lesser General Public License
%% along with TPIE.  If not, see <http:%%www.gnu.org/licenses/>

\chapter*{Acknowledgements}
\addcontentsline{toc}{chapter}{Acknowledgements}

The development of TPIE was supported in part by the National Science
Foundation under grants CCR-9007851 and EIA-9870734 and by the U.S.
Army Research Office under grants DAAL03-91-G-0035 and
DAAH04-96-1-0013.\comment{LA: Update before next distribution}

The authors would like to thank the following people for their
contributions to the development of TPIE; Jeff Vitter, Paul Natsev,
Eddie Grove, Roberto Tamassia, Yi-Jen Chiang, Mike Goodrich, Jyh-Jong
Tsay, Tom Cormen, Len Wisniewski, Liddy Shriver, David Kotz, Owen
Astrachan, Lipyeow Lim, Vasilis Samoladas, and Min Wang.

\comment{LA: There is some more text from Darren's ack. in here.}

%I would like to thank the following people for helpful discussions
%concerning algorithms and implementation techniques which influenced
%the development of TPIE: 
%\htmladdnormallink{Jeff Vitter}{%
%\begin{rawhtml}
%  http://www.cs.duke.edu/~jsv/HomePage.html
%\end{rawhtml}%
%}, 
%\index{Vitter, Jeff}
%\htmladdnormallink{Eddie Grove}{%
%\begin{rawhtml}
%  http://www.cs.duke.edu/cgi-bin/facinfo?efg
%\end{rawhtml}%
%}, 
%\index{Grove, Eddie}
%\htmladdnormallink{Roberto Tamassia}{%
%\begin{rawhtml}
%  http://www.cs.brown.edu/people/rt/
%\end{rawhtml}%
%}, 
%\index{Tamassia, Roberto}
%\htmladdnormallink{Yi-Jen Chiang}{%
%\begin{rawhtml}
%  http://www.cs.brown.edu/people/yjc/
%\end{rawhtml}%
%}, 
%\index{Chiang, Yi-Jen}
%\htmladdnormallink{Mike Goodrich}{%
%\begin{rawhtml}
%  http://www.cs.jhu.edu/goodrich/home.html
%\end{rawhtml}%
%}, 
%Jyh-Jong Tsay, 
%\index{Tsay, Jyh-Jong}
%\htmladdnormallink{Lars Arge}{%
%\begin{rawhtml}
%  http://www.daimi.aau.dk/~large/
%\end{rawhtml}%
%},
%\index{Arge, Lars}
%\htmladdnormallink{Tom Cormen}{%
%\begin{rawhtml}
%  http://www.cs.dartmouth.edu/faculty/cormen.html
%\end{rawhtml}%
%}, 
%\index{Cormen, Tom}
%Len Wisniewski, 
%\index{Wisniewski, Len}
%Liddy Shriver,
%\index{Shriver, Liddy}
%and 
%\htmladdnormallink{David Kotz}{%
%\begin{rawhtml}
%  http://www.cs.dartmouth.edu/faculty/kotz.html
%\end{rawhtml}%
%}.
%\index{Kotz, David}

%I would also like to thank the following people and institutions for
%providing access to the hardware on which TPIE design and development
%are ongoing: 
%\htmladdnormallink{Brown University Department of Computer
%  Science}{http://www.cs.brown.edu}, 
%\index{Brown University!Department of Computer Science}
%for Sun Sparc 10s running
%Solaris; 
%\htmladdnormallink{Duke University Department of Computer
%  Science}{http://www.cs.duke.edu}, 
%\index{Duke University!Department of Computer Science}
%for a variety of Sun workstations
%running SunOS and for DEC Alphas running OSF/1; 
%Yale Patt\index{Patt, Yale} and the ACAL Lab in the 
%\htmladdnormallink{Department of Electrical Engineering and
%  Computer Science}{http://www.eecs.umich.edu} at the University of
%Michigan, 
%\index{University of Michigan!ACAL Lab}
%for a DEC Alpha running OSF/1 and for an HP 9000 running
%HP-UX; 
%\htmladdnormallink{David
%  Kotz}{http://www.cs.dartmouth.edu/faculty/kotz.html} and the
%\htmladdnormallink{Dartmouth College Department of Computer
%  Science}{http://www.cs.dartmouth.edu},
%\index{Kotz, David}\index{Dartmouth College!Department of Computer Science}
% for MIPS based DECstations
%running Ultrix.  

%Finally, I would like to thank 
%\htmladdnormallink{Owen Astrachan}{http://www.cs.duke.edu/\~ola/HomePage.html}
%\index{Astrachan, Owen}
%for his helpful discussions on some of the finer points of the C++
%language.

}

\dedication{
\input{dedication.inc}
}

%%%%%%%%%%%%%%%%%%%%%%%%%%%%%%%%%%%%%%%%%%%%%%%%%%%%%%%%%%%%%\
%%%% Tweak float placements
% From: http://mintaka.sdsu.edu/GF/bibliog/latex/floats.html "Controlling LaTeX Floats"
% and based on: http://www.tex.ac.uk/cgi-bin/texfaq2html?label=floats
% LaTeX defaults listed at: http://people.cs.uu.nl/piet/floats/node1.html

% Alter some LaTeX defaults for better treatment of figures:
    % See p.105 of "TeX Unbound" for suggested values.
    % See pp. 199-200 of Lamport's "LaTeX" book for details.
    %   General parameters, for ALL pages:
    \renewcommand{\topfraction}{0.85}   % max fraction of floats at top
    \renewcommand{\bottomfraction}{0.6} % max fraction of floats at bottom
    %   Parameters for TEXT pages (not float pages):
    \setcounter{topnumber}{2}
    \setcounter{bottomnumber}{2}
    \setcounter{totalnumber}{4}     % 2 may work better
    \setcounter{dbltopnumber}{2}    % for 2-column pages
    \renewcommand{\dbltopfraction}{0.66}    % fit big float above 2-col. text
    \renewcommand{\textfraction}{0.15}  % allow minimal text w. figs
    %   Parameters for FLOAT pages (not text pages):
    \renewcommand{\floatpagefraction}{0.66} % require fuller float pages
    % N.B.: floatpagefraction MUST be less than topfraction !!
    \renewcommand{\dblfloatpagefraction}{0.66}  % require fuller float pages


%%%%%%%%%%%%%%%%%%%%%%%%%%%%%%%%%%%%%%%%%%%%%%%%%%%%%%%%%%%%%\
%%%% Use packages

%\usepackage{amsfonts}

%%% For figures
\usepackage{graphicx}
%\usepackage{subfig,rotate}

%%% for comments
\usepackage{verbatim}

% nice references
\usepackage[sort,numbers]{natbib}

% nice description style
\usepackage{enumitem}
\setlist[description]{style=nextline,
                      leftmargin=1.5em,
                      labelindent=1em,
                      topsep=1em,
                      itemsep=0em}

% for subfigure
\usepackage[margin=8pt]{subcaption}
\captionsetup{labelfont=bf}

% for code listings
\usepackage{listings}
\IfFileExists{inconsolata.sty}{\usepackage{inconsolata}}{\usepackage{zi4}}
\usepackage{color}
\definecolor{codebg}{RGB}{248,248,248} % mimics html code style
\definecolor{codeborder}{RGB}{204,204,204}
\lstset{breaklines=true,
        breakatwhitespace=false,
        basicstyle=\singlespacing\footnotesize\ttfamily,
        columns=fixed,
        frame=single,
        rulecolor=\color{codeborder},
        backgroundcolor=\color{codebg}
        }

%%% For tables
\usepackage{multirow}
% Longtable lets you have tables that span multiple pages.
\usepackage{longtable}

% Booktabs produces far nicer tables than the standard LaTeX tables.
%   see: http://en.wikibooks.org/wiki/LaTeX/Tables
\usepackage{booktabs}

% adds more formatting options to tables
\usepackage{array}

%set parameters for longtable:
% default caption width is 4in for longtable, but wider for normal tables
\setlength{\LTcapwidth}{\textwidth}

%%%%%%%%%%%%%%%%%%%%%%%%%%%%%%%%%%%%%%%%%%%%%%%%%%%%%%%%%%
%%% Printed vs. online formatting
\ifdefined\printmode

% Printed copy
% url package understands urls (with proper line-breaks) without hyperlinking them
\usepackage{url}


\else

\ifdefined\proquestmode
%ProQuest copy -- http://www.princeton.edu/~mudd/thesis/Submissionguide.pdf

% ProQuest requires a double spaced version (set previously). They will take an electronic copy, so we want links in the pdf, but also copies may be printed or made into microfilm in black and white, so we want outlined links instead of colored links.
\usepackage{hyperref}
\hypersetup{bookmarksnumbered}

% copy the already-set title and author to use in the pdf properties
\makeatletter
\hypersetup{pdftitle=\@title,pdfauthor=\@author}
\makeatother

\else
% Online copy

% adds internal linked references, pdf bookmarks, etc

% turn all references and citations into hyperlinks:
%  -- not for printed copies
% -- automatically includes url package
% options:
%   colorlinks makes links by coloring the text instead of putting a rectangle around the text.
\usepackage{hyperref}
\hypersetup{colorlinks,bookmarksnumbered}

% copy the already-set title and author to use in the pdf properties
\makeatletter
\hypersetup{pdftitle=\@title,pdfauthor=\@author}
\makeatother

% make the page number rather than the text be the link for ToC entries
%\hypersetup{linktocpage}
\fi % proquest or online formatting
\fi % printed or online formatting
